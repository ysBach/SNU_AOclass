\chapter{Introduction}

In math, people prefer to use $ \log_e $ or just $ \log $ rather than $ \ln $. But due to its simplicity (especially when typing), I will stick to the usage of $ \ln $ which is identical to $ \log_e $. But to avoid any confusion, I will always specify the base 10 ($ \lg $).

In astronomy, people use $ \equiv $ as ``is defined as'' or any kind of ``trivial by the definition'' cases. But I will use $ := $ for the definition and use $ \equiv $ to mean ``equality holds trivially by the definition''. For example, under 1-D linear homogeneous isotropic matter, linear electric polarization is written as $ P \equiv \epsilon_0 \chi E $ or $ \chi := P/\epsilon_0 E $. The latter is the \textit{definition} of electric susceptability, while the former is \textit{trivially true by the definition}. 

In this note, I used some math-like notations like theorem (Thm). However, the theorem here is not necessarily the same as that of mathematics. I tried to put some important assumptions and summary notes to theorem

The reference ``Walpole'' is Walpole et al. 2013, Essentials of Probabilities and Statistics.