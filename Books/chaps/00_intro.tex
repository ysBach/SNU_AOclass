\chapter*{Introduction}

In math, people prefer to use ``$ \log_e $'' or just ``$ \log $'' rather than ``$ \ln $'', which is the opposite to many scientific fields. Due to its simplicity (especially when typing), I will also stick to the usage of ``$ \ln $'' which is identical to ``$ \log_e $''. But to avoid any confusion, I will always specify the base 10 (``$ \lg $'').

At many places in physics and astronomy, people use $ \equiv $ as ``is defined as'' or any kind of ``trivial by the definition'' cases. But I will use $ := $ for the definition and use $ \equiv $ to mean ``equality holds trivially by the definition''. For example, under 1-D linear homogeneous isotropic matter, linear electric polarization is written as $ P \equiv \epsilon_0 \chi E $ or $ \chi := P/\epsilon_0 E $. The latter is the \textit{definition} of electric susceptability, while the former is \textit{trivially true by the definition}. 

For distributions, $ \sim $ means ``follows'', i.e., $ X \sim \mathcal{N}(\mu, \sigma^2) $ means ``the random variable $ X $ follows the Gaussian (normal) distribution with mean $ \mu $ and standard deviation $ \sigma $ (thus variance $ \sigma^2 $).'' When this is an approximation, the symbol with dot ``$ \simdot $'' is used, such as $ X \simdot \mathcal{N}(\mu, \sigma^2) $. In this case, it is interpreted as ``approximately follows.''

In this note, I used some math-like notations like theorem (Thm). However, the theorem here is not necessarily the same as that of mathematics. I tried to put some important assumptions and summary notes to theorem.

The references are 
\begin{itemize}
\item ``Walpole'' is Walpole et al. 2013, ``Essentials of Probabilities and Statistics''.
\item ``Sivia'' is Sivia and Skilling 2006, ``Data Analysis A Bayesian Tutorial'', 2/e.
\end{itemize}